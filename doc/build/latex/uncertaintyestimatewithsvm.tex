%% Generated by Sphinx.
\def\sphinxdocclass{report}
\documentclass[a4paper,10pt,english]{sphinxmanual}
\ifdefined\pdfpxdimen
   \let\sphinxpxdimen\pdfpxdimen\else\newdimen\sphinxpxdimen
\fi \sphinxpxdimen=.75bp\relax

\PassOptionsToPackage{warn}{textcomp}
\usepackage[utf8]{inputenc}
\ifdefined\DeclareUnicodeCharacter
% support both utf8 and utf8x syntaxes
  \ifdefined\DeclareUnicodeCharacterAsOptional
    \def\sphinxDUC#1{\DeclareUnicodeCharacter{"#1}}
  \else
    \let\sphinxDUC\DeclareUnicodeCharacter
  \fi
  \sphinxDUC{00A0}{\nobreakspace}
  \sphinxDUC{2500}{\sphinxunichar{2500}}
  \sphinxDUC{2502}{\sphinxunichar{2502}}
  \sphinxDUC{2514}{\sphinxunichar{2514}}
  \sphinxDUC{251C}{\sphinxunichar{251C}}
  \sphinxDUC{2572}{\textbackslash}
\fi
\usepackage{cmap}
\usepackage[T1]{fontenc}
\usepackage{amsmath,amssymb,amstext}
\usepackage{babel}



\usepackage{times}
\expandafter\ifx\csname T@LGR\endcsname\relax
\else
% LGR was declared as font encoding
  \substitutefont{LGR}{\rmdefault}{cmr}
  \substitutefont{LGR}{\sfdefault}{cmss}
  \substitutefont{LGR}{\ttdefault}{cmtt}
\fi
\expandafter\ifx\csname T@X2\endcsname\relax
  \expandafter\ifx\csname T@T2A\endcsname\relax
  \else
  % T2A was declared as font encoding
    \substitutefont{T2A}{\rmdefault}{cmr}
    \substitutefont{T2A}{\sfdefault}{cmss}
    \substitutefont{T2A}{\ttdefault}{cmtt}
  \fi
\else
% X2 was declared as font encoding
  \substitutefont{X2}{\rmdefault}{cmr}
  \substitutefont{X2}{\sfdefault}{cmss}
  \substitutefont{X2}{\ttdefault}{cmtt}
\fi


\usepackage[Sonny]{fncychap}
\ChNameVar{\Large\normalfont\sffamily}
\ChTitleVar{\Large\normalfont\sffamily}
\usepackage{sphinx}

\fvset{fontsize=\small}
\usepackage{geometry}

% Include hyperref last.
\usepackage{hyperref}
% Fix anchor placement for figures with captions.
\usepackage{hypcap}% it must be loaded after hyperref.
% Set up styles of URL: it should be placed after hyperref.
\urlstyle{same}
\addto\captionsenglish{\renewcommand{\contentsname}{Contents:}}

\usepackage{sphinxmessages}
\setcounter{tocdepth}{1}



\title{Uncertainty Estimate with SVM}
\date{Dec 01, 2019}
\release{0.0.1}
\author{Marko Djordjic}
\newcommand{\sphinxlogo}{\vbox{}}
\renewcommand{\releasename}{Release}
\makeindex
\begin{document}

\ifdefined\shorthandoff
  \ifnum\catcode`\=\string=\active\shorthandoff{=}\fi
  \ifnum\catcode`\"=\active\shorthandoff{"}\fi
\fi

\pagestyle{empty}
\sphinxmaketitle
\pagestyle{plain}
\sphinxtableofcontents
\pagestyle{normal}
\phantomsection\label{\detokenize{index::doc}}



\chapter{ucf Module}
\label{\detokenize{index:module-ucf}}\label{\detokenize{index:ucf-module}}\index{ucf (module)@\spxentry{ucf}\spxextra{module}}
Uncertain estimate with SVM is a package which provides a functional
example on how to train a SVC which outputs an uncertainty estimate
alongside prediction of class membership.

As noted by Murphy (2012, p. 497) are not probabilistic models. The

In this specific case, a classification task is solved with a SVC. The
same methodology can be extended to solving regression problems as well.
In the case of solving a classification task, a specific approach towards
building a training set has been applied. Each of the estimators is trained
on a subset of training data set. This subset is in itself designed to
contain all classes in equal proportions. Therefore the size of the subset
is determined by the size of least frequent class and the number of classes.

In order to obtain uncertainty estimate from an otherwise non-probabilistic
model, a variational inference approach was utilized.
\subsubsection*{Examples}

\begin{sphinxVerbatim}[commandchars=\\\{\}]
\PYG{g+gp}{\PYGZgt{}\PYGZgt{}\PYGZgt{} }\PYG{k+kn}{import} \PYG{n+nn}{uncertainty\PYGZus{}estimate\PYGZus{}with\PYGZus{}svm}\PYG{n+nn}{.}\PYG{n+nn}{ucf} \PYG{k}{as} \PYG{n+nn}{ue\PYGZus{}svm}
\PYG{g+gp}{\PYGZgt{}\PYGZgt{}\PYGZgt{} }\PYG{c+c1}{\PYGZsh{} Get the training set with the equal number of classes.}
\PYG{g+gp}{\PYGZgt{}\PYGZgt{}\PYGZgt{} }\PYG{n}{reduced\PYGZus{}set} \PYG{o}{=} \PYG{n}{ue\PYGZus{}svm}\PYG{o}{.}\PYG{n}{reduce\PYGZus{}set\PYGZus{}to\PYGZus{}equal\PYGZus{}distribution\PYGZus{}of\PYGZus{}classes}\PYG{p}{(}
\PYG{g+go}{        features\PYGZus{}for\PYGZus{}training=features, targets\PYGZus{}for\PYGZus{}training=targets}
\PYG{g+go}{        )}
\PYG{g+gp}{\PYGZgt{}\PYGZgt{}\PYGZgt{} }\PYG{n}{ensemble} \PYG{o}{=} \PYG{n}{ue\PYGZus{}svm}\PYG{o}{.}\PYG{n}{generate\PYGZus{}ensemble}\PYG{p}{(}\PYG{n}{number\PYGZus{}of\PYGZus{}estimators}\PYG{o}{=}\PYG{l+m+mi}{30}\PYG{p}{,}
\PYG{g+go}{        features\PYGZus{}for\PYGZus{}training=reduced\PYGZus{}set[, 0:9],}
\PYG{g+go}{        targets\PYGZus{}for\PYGZus{}training=reduced\PYGZus{}set[, 10]}
\PYG{g+go}{        )}
\PYG{g+gp}{\PYGZgt{}\PYGZgt{}\PYGZgt{} }\PYG{n}{predictions}\PYG{p}{,} \PYG{n}{uncertainty} \PYG{o}{=} \PYG{n}{ue\PYGZus{}svm}\PYG{o}{.}\PYG{n}{generate\PYGZus{}predictions}\PYG{p}{(}
\PYG{g+go}{        inventory\PYGZus{}of\PYGZus{}estimators=ensemble,}
\PYG{g+go}{        features=x\PYGZus{}test}
\PYG{g+go}{    )}
\end{sphinxVerbatim}

\newpage

Below we can take a look at the solution produced by the single SVM.

\begin{figure}[H]
\centering

\noindent\sphinxincludegraphics{{010_single_svc_predictions}.jpg}
\end{figure}

And here is the confusion matrix for the single SVM.

\begin{figure}[H]
\centering

\noindent\sphinxincludegraphics{{010_single_svc_cn_matrix}.jpg}
\end{figure}

\newpage

SVC ensemble produces different solution, as we can see from the image
below.

\begin{figure}[H]
\centering

\noindent\sphinxincludegraphics{{010_ensemble_svm_prediction}.jpg}
\end{figure}

Also distribution of classes across the prediction is different.

\begin{figure}[H]
\centering

\noindent\sphinxincludegraphics{{010_ensemble_svm_cn_matrix}.jpg}
\end{figure}

\newpage

Finally we can take a look a te comparison between a single SVC and an
ensemble of SVCs.

\begin{figure}[H]
\centering

\noindent\sphinxincludegraphics{{010_comparison}.jpg}
\end{figure}

There is clearly a different fit, which has been achieved on the basis
of training of multiple SVCs on a balanced set, and some of the bias
of the model trained on the imbalanced set has be removed.
However, even though this allows for training of SVMs on larger training
sets, there are two relative problems with this solution: (a) minor
but still present loss in accuracy, and (b) slower execution
of the ensemble.
\subsubsection*{References}
\subsubsection*{Notes}


\section{Functions}
\label{\detokenize{index:functions}}

\begin{savenotes}\sphinxatlongtablestart\begin{longtable}[c]{\X{1}{2}\X{1}{2}}
\hline

\endfirsthead

\multicolumn{2}{c}%
{\makebox[0pt]{\sphinxtablecontinued{\tablename\ \thetable{} -- continued from previous page}}}\\
\hline

\endhead

\hline
\multicolumn{2}{r}{\makebox[0pt][r]{\sphinxtablecontinued{Continued on next page}}}\\
\endfoot

\endlastfoot

{\hyperref[\detokenize{api/ucf.compute_predictive_entropy:ucf.compute_predictive_entropy}]{\sphinxcrossref{\sphinxcode{\sphinxupquote{compute\_predictive\_entropy}}}}}(probability)
&
Estimate epistemic uncertainty via predictive entropy {\color{red}\bfseries{}{[}1{]}\_}
\\
\hline
{\hyperref[\detokenize{api/ucf.generate_ensemble:ucf.generate_ensemble}]{\sphinxcrossref{\sphinxcode{\sphinxupquote{generate\_ensemble}}}}}(number\_of\_estimators, ...)
&
Generates a collection of estimators
\\
\hline
{\hyperref[\detokenize{api/ucf.generate_predictions:ucf.generate_predictions}]{\sphinxcrossref{\sphinxcode{\sphinxupquote{generate\_predictions}}}}}(...)
&
Generate predictions from ensemble
\\
\hline
{\hyperref[\detokenize{api/ucf.get_all_files_within_folder:ucf.get_all_files_within_folder}]{\sphinxcrossref{\sphinxcode{\sphinxupquote{get\_all\_files\_within\_folder}}}}}(path, ...)
&
Catalogue all files within a folder according to negative condition
\\
\hline
{\hyperref[\detokenize{api/ucf.get_all_sub_folders_within_folder:ucf.get_all_sub_folders_within_folder}]{\sphinxcrossref{\sphinxcode{\sphinxupquote{get\_all\_sub\_folders\_within\_folder}}}}}(path)
&
Catalogue names of all sub-folders within folder
\\
\hline
{\hyperref[\detokenize{api/ucf.get_data:ucf.get_data}]{\sphinxcrossref{\sphinxcode{\sphinxupquote{get\_data}}}}}(folder\_wh\_data)
&
Build an inventory of data sets out of individual files within a folder.
\\
\hline
{\hyperref[\detokenize{api/ucf.make_confusion_matrix:ucf.make_confusion_matrix}]{\sphinxcrossref{\sphinxcode{\sphinxupquote{make\_confusion\_matrix}}}}}(reference, output, ...)
&
Compute confusion matrix
\\
\hline
{\hyperref[\detokenize{api/ucf.plot_comparison:ucf.plot_comparison}]{\sphinxcrossref{\sphinxcode{\sphinxupquote{plot\_comparison}}}}}(coordinates, reference, ...)
&
Plot comparison across different classification solutions.
\\
\hline
{\hyperref[\detokenize{api/ucf.plot_confusion_matrix:ucf.plot_confusion_matrix}]{\sphinxcrossref{\sphinxcode{\sphinxupquote{plot\_confusion\_matrix}}}}}(content{[}, save\_plot, path{]})
&
Plot confusion matrix
\\
\hline
{\hyperref[\detokenize{api/ucf.plot_individual_classes:ucf.plot_individual_classes}]{\sphinxcrossref{\sphinxcode{\sphinxupquote{plot\_individual\_classes}}}}}(coordinates, ...{[}, ...{]})
&
Plot class membership in individual plot
\\
\hline
{\hyperref[\detokenize{api/ucf.plot_solution:ucf.plot_solution}]{\sphinxcrossref{\sphinxcode{\sphinxupquote{plot\_solution}}}}}(coordinates, original\_labels, ...)
&
Plotting of solution of classification task
\\
\hline
{\hyperref[\detokenize{api/ucf.reduce_set_to_equal_distribution_of_classes:ucf.reduce_set_to_equal_distribution_of_classes}]{\sphinxcrossref{\sphinxcode{\sphinxupquote{reduce\_set\_to\_equal\_distribution\_of\_classes}}}}}(...)
&
Reduces the training set to the size N = K x size of the least frequent class
\\
\hline
{\hyperref[\detokenize{api/ucf.scatter_plot_with_groups:ucf.scatter_plot_with_groups}]{\sphinxcrossref{\sphinxcode{\sphinxupquote{scatter\_plot\_with\_groups}}}}}(coordinates, ...{[}, ...{]})
&
Produce scatter plot with coloration according to the labels
\\
\hline
\end{longtable}\sphinxatlongtableend\end{savenotes}


\subsection{compute\_predictive\_entropy}
\label{\detokenize{api/ucf.compute_predictive_entropy:compute-predictive-entropy}}\label{\detokenize{api/ucf.compute_predictive_entropy::doc}}\index{compute\_predictive\_entropy() (in module ucf)@\spxentry{compute\_predictive\_entropy()}\spxextra{in module ucf}}

\begin{fulllineitems}
\phantomsection\label{\detokenize{api/ucf.compute_predictive_entropy:ucf.compute_predictive_entropy}}\pysiglinewithargsret{\sphinxcode{\sphinxupquote{ucf.}}\sphinxbfcode{\sphinxupquote{compute\_predictive\_entropy}}}{\emph{probability}}{}
Estimate epistemic uncertainty via predictive entropy %
\begin{footnote}[1]\sphinxAtStartFootnote
Further details about about predictive entropy available at:
\sphinxtitleref{https://en.wikipedia.org/wiki/Entropy\_(information\_theory)}
%
\end{footnote}
\begin{quote}\begin{description}
\item[{Parameters}] \leavevmode
\sphinxstyleliteralstrong{\sphinxupquote{probability}} (\sphinxstyleliteralemphasis{\sphinxupquote{numpy.array}}) -- A numpy.array (N x C) with the probabilities obtained from the
underlying classier (soft voting).

\item[{Returns}] \leavevmode
Uncertainty estimate for each prediction.

\item[{Return type}] \leavevmode
numpy.array

\end{description}\end{quote}
\subsubsection*{Notes}

For the computation of uncertainty value equal to zero are
replaced with a small constant near zero.
\subsubsection*{References}

\end{fulllineitems}



\subsection{generate\_ensemble}
\label{\detokenize{api/ucf.generate_ensemble:generate-ensemble}}\label{\detokenize{api/ucf.generate_ensemble::doc}}\index{generate\_ensemble() (in module ucf)@\spxentry{generate\_ensemble()}\spxextra{in module ucf}}

\begin{fulllineitems}
\phantomsection\label{\detokenize{api/ucf.generate_ensemble:ucf.generate_ensemble}}\pysiglinewithargsret{\sphinxcode{\sphinxupquote{ucf.}}\sphinxbfcode{\sphinxupquote{generate\_ensemble}}}{\emph{number\_of\_estimators}, \emph{features\_for\_training}, \emph{targets\_for\_training}}{}
Generates a collection of estimators

Each estimator is trained on a sub-set of the training data, and
appended to the ensemble. Support Vector Classifier has been
selected as the classifier of choice, but can be replaced with any
other classifier.
\begin{quote}\begin{description}
\item[{Parameters}] \leavevmode\begin{itemize}
\item {} 
\sphinxstyleliteralstrong{\sphinxupquote{number\_of\_estimators}} (\sphinxstyleliteralemphasis{\sphinxupquote{int}}) -- How much estimators will be in the ensemble.

\item {} 
\sphinxstyleliteralstrong{\sphinxupquote{features\_for\_training}} (\sphinxstyleliteralemphasis{\sphinxupquote{numpy.array}}) -- Features which will be utilized for training of individual
estimators.

\item {} 
\sphinxstyleliteralstrong{\sphinxupquote{targets\_for\_training}} (\sphinxstyleliteralemphasis{\sphinxupquote{numpy.array}}) -- Targets which will be utilized for training of individual
estimators.

\end{itemize}

\item[{Returns}] \leavevmode
A collection of SVCs trained on different sections of features
and targets pairs.

\item[{Return type}] \leavevmode
List

\end{description}\end{quote}
\subsubsection*{Notes}

The function does not shuffle the data. If shuffling is necessary,
it has to be done before call to the function.

\end{fulllineitems}



\subsection{generate\_predictions}
\label{\detokenize{api/ucf.generate_predictions:generate-predictions}}\label{\detokenize{api/ucf.generate_predictions::doc}}\index{generate\_predictions() (in module ucf)@\spxentry{generate\_predictions()}\spxextra{in module ucf}}

\begin{fulllineitems}
\phantomsection\label{\detokenize{api/ucf.generate_predictions:ucf.generate_predictions}}\pysiglinewithargsret{\sphinxcode{\sphinxupquote{ucf.}}\sphinxbfcode{\sphinxupquote{generate\_predictions}}}{\emph{inventory\_of\_estimators}, \emph{features}}{}
Generate predictions from ensemble

The function applies 'predict\_proba' method to a collection
of estimators, in order to get predictions and compute uncertainty
estimate via predictive entropy.
\begin{quote}\begin{description}
\item[{Parameters}] \leavevmode\begin{itemize}
\item {} 
\sphinxstyleliteralstrong{\sphinxupquote{inventory\_of\_estimators}} (\sphinxstyleliteralemphasis{\sphinxupquote{list}}) -- A collection of estimators placed in a list.

\item {} 
\sphinxstyleliteralstrong{\sphinxupquote{features}} (\sphinxstyleliteralemphasis{\sphinxupquote{numpy.array}}) -- Features on which to perform prediction.

\end{itemize}

\item[{Returns}] \leavevmode
\begin{itemize}
\item {} 
\sphinxstylestrong{ensemble\_predictions} (\sphinxstyleemphasis{numpy.array}) -- Prediction of class membership.

\item {} 
\sphinxstylestrong{uncertainty\_estimate} (\sphinxstyleemphasis{numpy.array}) -- Uncertainty estimate.

\end{itemize}


\end{description}\end{quote}

\end{fulllineitems}



\subsection{get\_all\_files\_within\_folder}
\label{\detokenize{api/ucf.get_all_files_within_folder:get-all-files-within-folder}}\label{\detokenize{api/ucf.get_all_files_within_folder::doc}}\index{get\_all\_files\_within\_folder() (in module ucf)@\spxentry{get\_all\_files\_within\_folder()}\spxextra{in module ucf}}

\begin{fulllineitems}
\phantomsection\label{\detokenize{api/ucf.get_all_files_within_folder:ucf.get_all_files_within_folder}}\pysiglinewithargsret{\sphinxcode{\sphinxupquote{ucf.}}\sphinxbfcode{\sphinxupquote{get\_all\_files\_within\_folder}}}{\emph{path}, \emph{negative\_condition}}{}
Catalogue all files within a folder according to negative condition
\begin{quote}\begin{description}
\item[{Parameters}] \leavevmode\begin{itemize}
\item {} 
\sphinxstyleliteralstrong{\sphinxupquote{path}} (\sphinxstyleliteralemphasis{\sphinxupquote{str}}) -- Path to the folder.

\item {} 
\sphinxstyleliteralstrong{\sphinxupquote{negative\_condition}} (\sphinxstyleliteralemphasis{\sphinxupquote{str}}) -- Exact text contained with the name of the files, which is
utilized to identify files which will not be catalogued.

\end{itemize}

\item[{Returns}] \leavevmode
Names of all files within folder except file names designated in
\sphinxtitleref{negative\_condition} parameter.

\item[{Return type}] \leavevmode
list

\end{description}\end{quote}

\end{fulllineitems}



\subsection{get\_all\_sub\_folders\_within\_folder}
\label{\detokenize{api/ucf.get_all_sub_folders_within_folder:get-all-sub-folders-within-folder}}\label{\detokenize{api/ucf.get_all_sub_folders_within_folder::doc}}\index{get\_all\_sub\_folders\_within\_folder() (in module ucf)@\spxentry{get\_all\_sub\_folders\_within\_folder()}\spxextra{in module ucf}}

\begin{fulllineitems}
\phantomsection\label{\detokenize{api/ucf.get_all_sub_folders_within_folder:ucf.get_all_sub_folders_within_folder}}\pysiglinewithargsret{\sphinxcode{\sphinxupquote{ucf.}}\sphinxbfcode{\sphinxupquote{get\_all\_sub\_folders\_within\_folder}}}{\emph{path}}{}
Catalogue names of all sub-folders within folder
\begin{quote}\begin{description}
\item[{Parameters}] \leavevmode
\sphinxstyleliteralstrong{\sphinxupquote{path}} (\sphinxstyleliteralemphasis{\sphinxupquote{str}}) -- Path to the main folder.

\item[{Returns}] \leavevmode
Names of all sub-folder within folder designated in \sphinxtitleref{path} parameter.

\item[{Return type}] \leavevmode
list

\end{description}\end{quote}

\end{fulllineitems}



\subsection{get\_data}
\label{\detokenize{api/ucf.get_data:get-data}}\label{\detokenize{api/ucf.get_data::doc}}\index{get\_data() (in module ucf)@\spxentry{get\_data()}\spxextra{in module ucf}}

\begin{fulllineitems}
\phantomsection\label{\detokenize{api/ucf.get_data:ucf.get_data}}\pysiglinewithargsret{\sphinxcode{\sphinxupquote{ucf.}}\sphinxbfcode{\sphinxupquote{get\_data}}}{\emph{folder\_wh\_data}}{}
Build an inventory of data sets out of individual files within a
folder.

Function omits files which are having \sphinxtitleref{txt} inside their name.
\begin{quote}\begin{description}
\item[{Parameters}] \leavevmode
\sphinxstyleliteralstrong{\sphinxupquote{folder\_wh\_data}} (\sphinxstyleliteralemphasis{\sphinxupquote{str}}) -- Folder in which data files are residing.

\item[{Returns}] \leavevmode
All the data sets generated from individual data
files.

\item[{Return type}] \leavevmode
list

\end{description}\end{quote}

\end{fulllineitems}



\subsection{make\_confusion\_matrix}
\label{\detokenize{api/ucf.make_confusion_matrix:make-confusion-matrix}}\label{\detokenize{api/ucf.make_confusion_matrix::doc}}\index{make\_confusion\_matrix() (in module ucf)@\spxentry{make\_confusion\_matrix()}\spxextra{in module ucf}}

\begin{fulllineitems}
\phantomsection\label{\detokenize{api/ucf.make_confusion_matrix:ucf.make_confusion_matrix}}\pysiglinewithargsret{\sphinxcode{\sphinxupquote{ucf.}}\sphinxbfcode{\sphinxupquote{make\_confusion\_matrix}}}{\emph{reference}, \emph{output}, \emph{prediction\_labels}}{}
Compute confusion matrix
\begin{quote}\begin{description}
\item[{Parameters}] \leavevmode\begin{itemize}
\item {} 
\sphinxstyleliteralstrong{\sphinxupquote{reference}} (\sphinxstyleliteralemphasis{\sphinxupquote{numpy.array}}) -- A vector with reference.

\item {} 
\sphinxstyleliteralstrong{\sphinxupquote{output}} (\sphinxstyleliteralemphasis{\sphinxupquote{numpy.array}}) -- A vector with targets.

\item {} 
\sphinxstyleliteralstrong{\sphinxupquote{prediction\_labels}} (\sphinxstyleliteralemphasis{\sphinxupquote{list}}) -- Descriptions of labels.

\end{itemize}

\item[{Returns}] \leavevmode


\item[{Return type}] \leavevmode
Confusion matrix as the pandas.DataFrame.

\end{description}\end{quote}
\subsubsection*{Notes}

Reference group is placed in row. Proportion of each prediction
within the reference group is computed across columns (horizontally).

\end{fulllineitems}



\subsection{plot\_comparison}
\label{\detokenize{api/ucf.plot_comparison:plot-comparison}}\label{\detokenize{api/ucf.plot_comparison::doc}}\index{plot\_comparison() (in module ucf)@\spxentry{plot\_comparison()}\spxextra{in module ucf}}

\begin{fulllineitems}
\phantomsection\label{\detokenize{api/ucf.plot_comparison:ucf.plot_comparison}}\pysiglinewithargsret{\sphinxcode{\sphinxupquote{ucf.}}\sphinxbfcode{\sphinxupquote{plot\_comparison}}}{\emph{coordinates}, \emph{reference}, \emph{solutions}, \emph{description}, \emph{coloration}, \emph{save=True}, \emph{path=None}}{}
Plot comparison across different classification solutions.
\begin{quote}\begin{description}
\item[{Parameters}] \leavevmode\begin{itemize}
\item {} 
\sphinxstyleliteralstrong{\sphinxupquote{coordinates}} (\sphinxstyleliteralemphasis{\sphinxupquote{numpy.array}}) -- Coordinates of the points.

\item {} 
\sphinxstyleliteralstrong{\sphinxupquote{reference}} (\sphinxstyleliteralemphasis{\sphinxupquote{numpy.array}}) -- Labels of the reference.

\item {} 
\sphinxstyleliteralstrong{\sphinxupquote{solutions}} (\sphinxstyleliteralemphasis{\sphinxupquote{list}}) -- List containing solutions of classification problem.

\item {} 
\sphinxstyleliteralstrong{\sphinxupquote{description}} (\sphinxstyleliteralemphasis{\sphinxupquote{dict}}) -- Description of each label.

\item {} 
\sphinxstyleliteralstrong{\sphinxupquote{coloration}} (\sphinxstyleliteralemphasis{\sphinxupquote{dict}}) -- Vector with coloration.

\end{itemize}

\item[{Returns}] \leavevmode
No explicit return. Plot is displayed on the screen, or saved
into a file.

\item[{Return type}] \leavevmode
None

\end{description}\end{quote}

\end{fulllineitems}



\subsection{plot\_confusion\_matrix}
\label{\detokenize{api/ucf.plot_confusion_matrix:plot-confusion-matrix}}\label{\detokenize{api/ucf.plot_confusion_matrix::doc}}\index{plot\_confusion\_matrix() (in module ucf)@\spxentry{plot\_confusion\_matrix()}\spxextra{in module ucf}}

\begin{fulllineitems}
\phantomsection\label{\detokenize{api/ucf.plot_confusion_matrix:ucf.plot_confusion_matrix}}\pysiglinewithargsret{\sphinxcode{\sphinxupquote{ucf.}}\sphinxbfcode{\sphinxupquote{plot\_confusion\_matrix}}}{\emph{content}, \emph{save\_plot=False}, \emph{path=None}}{}
Plot confusion matrix

Confusion matrix will always be tabulated and plotted. Optionally,
picture of confusion matrix can be saved.
\begin{quote}\begin{description}
\item[{Parameters}] \leavevmode\begin{itemize}
\item {} 
\sphinxstyleliteralstrong{\sphinxupquote{content}} (\sphinxstyleliteralemphasis{\sphinxupquote{numpy.array}}) -- Numpy array with the complete content of the confusion matrix.

\item {} 
\sphinxstyleliteralstrong{\sphinxupquote{save\_plot}} (\sphinxstyleliteralemphasis{\sphinxupquote{bool}}) -- Indication whether to save the plot. Default set to false.

\item {} 
\sphinxstyleliteralstrong{\sphinxupquote{path}} (\sphinxstyleliteralemphasis{\sphinxupquote{str}}) -- Path including the file name where to save the plot.

\end{itemize}

\item[{Returns}] \leavevmode
No explicit return. Optionally plot can be saved.

\item[{Return type}] \leavevmode
None

\end{description}\end{quote}
\subsubsection*{Notes}

Content must consist all numeric content, as well as headings of
all rows and columns.

\end{fulllineitems}



\subsection{plot\_individual\_classes}
\label{\detokenize{api/ucf.plot_individual_classes:plot-individual-classes}}\label{\detokenize{api/ucf.plot_individual_classes::doc}}\index{plot\_individual\_classes() (in module ucf)@\spxentry{plot\_individual\_classes()}\spxextra{in module ucf}}

\begin{fulllineitems}
\phantomsection\label{\detokenize{api/ucf.plot_individual_classes:ucf.plot_individual_classes}}\pysiglinewithargsret{\sphinxcode{\sphinxupquote{ucf.}}\sphinxbfcode{\sphinxupquote{plot\_individual\_classes}}}{\emph{coordinates}, \emph{class\_membership}, \emph{description}, \emph{coloration\_mode}, \emph{coloration}, \emph{save=False}, \emph{path=None}}{}
Plot class membership in individual plot

Point within scatter plots indicating class membership with multiple
classes can often overlap, therefore debilitating correct analysis. This
function plots all classes independently.
\begin{quote}\begin{description}
\item[{Parameters}] \leavevmode\begin{itemize}
\item {} 
\sphinxstyleliteralstrong{\sphinxupquote{coordinates}} (\sphinxstyleliteralemphasis{\sphinxupquote{numpy.array}}) -- Coordinates of points.

\item {} 
\sphinxstyleliteralstrong{\sphinxupquote{class\_membership}} (\sphinxstyleliteralemphasis{\sphinxupquote{numpy.array}}) -- Indication of class membership.

\item {} 
\sphinxstyleliteralstrong{\sphinxupquote{description}} (\sphinxstyleliteralemphasis{\sphinxupquote{dict}}) -- Description of each label.

\item {} 
\sphinxstyleliteralstrong{\sphinxupquote{coloration\_mode}} (\sphinxstyleliteralemphasis{\sphinxupquote{str}}) -- Indication of the mode of coloration.

\item {} 
\sphinxstyleliteralstrong{\sphinxupquote{coloration}} (\sphinxstyleliteralemphasis{\sphinxupquote{numpy.array}}) -- Vector with coloration.

\item {} 
\sphinxstyleliteralstrong{\sphinxupquote{save}} (\sphinxstyleliteralemphasis{\sphinxupquote{bool}}) -- Option to save the plot.

\item {} 
\sphinxstyleliteralstrong{\sphinxupquote{path}} (\sphinxstyleliteralemphasis{\sphinxupquote{str}}) -- Absolute path towards the file in which to save the plot.

\end{itemize}

\item[{Returns}] \leavevmode
No explicit return.

\item[{Return type}] \leavevmode
None

\end{description}\end{quote}

\end{fulllineitems}



\subsection{plot\_solution}
\label{\detokenize{api/ucf.plot_solution:plot-solution}}\label{\detokenize{api/ucf.plot_solution::doc}}\index{plot\_solution() (in module ucf)@\spxentry{plot\_solution()}\spxextra{in module ucf}}

\begin{fulllineitems}
\phantomsection\label{\detokenize{api/ucf.plot_solution:ucf.plot_solution}}\pysiglinewithargsret{\sphinxcode{\sphinxupquote{ucf.}}\sphinxbfcode{\sphinxupquote{plot\_solution}}}{\emph{coordinates}, \emph{original\_labels}, \emph{predicted\_labels}, \emph{legend\_colors}, \emph{legend\_descriptions}, \emph{uncertainty}, \emph{save=False}, \emph{path=None}}{}
Plotting of solution of classification task

Convenience function to plot: (a) original labels, (b) predicted
labels, and (c) uncertainty estimate of the model.
\begin{quote}\begin{description}
\item[{Parameters}] \leavevmode\begin{itemize}
\item {} 
\sphinxstyleliteralstrong{\sphinxupquote{coordinates}} (\sphinxstyleliteralemphasis{\sphinxupquote{numpy.array}}) -- Coordinates of labels

\item {} 
\sphinxstyleliteralstrong{\sphinxupquote{original\_labels}} (\sphinxstyleliteralemphasis{\sphinxupquote{numpy.array}}) -- Reference one-dimensional encoding of the class membership.
One-hot encoding is not supported.

\item {} 
\sphinxstyleliteralstrong{\sphinxupquote{predicted\_labels}} (\sphinxstyleliteralemphasis{\sphinxupquote{numpy.array}}) -- Predicted one-dimensional encoding of the class membership.
One-hot encoding is not supported.

\item {} 
\sphinxstyleliteralstrong{\sphinxupquote{legend\_colors}} (\sphinxstyleliteralemphasis{\sphinxupquote{dict}}) -- Colors to be utilized for coloration of points.

\item {} 
\sphinxstyleliteralstrong{\sphinxupquote{legend\_descriptions}} (\sphinxstyleliteralemphasis{\sphinxupquote{dict}}) -- Labels to be utilized for description in plot legend.

\item {} 
\sphinxstyleliteralstrong{\sphinxupquote{uncertainty}} (\sphinxstyleliteralemphasis{\sphinxupquote{numpy.array}}) -- Uncertainty of the models estimate of class membership.

\item {} 
\sphinxstyleliteralstrong{\sphinxupquote{save}} (\sphinxstyleliteralemphasis{\sphinxupquote{bool}}) -- Option to save the plot. Default set to false.

\item {} 
\sphinxstyleliteralstrong{\sphinxupquote{path}} (\sphinxstyleliteralemphasis{\sphinxupquote{str}}) -- Absolute path to the file in which to save a plot.

\end{itemize}

\item[{Returns}] \leavevmode
No explicit return. Plot is displayed on the screen, or saved
into a file.

\item[{Return type}] \leavevmode
None

\end{description}\end{quote}

\end{fulllineitems}



\subsection{reduce\_set\_to\_equal\_distribution\_of\_classes}
\label{\detokenize{api/ucf.reduce_set_to_equal_distribution_of_classes:reduce-set-to-equal-distribution-of-classes}}\label{\detokenize{api/ucf.reduce_set_to_equal_distribution_of_classes::doc}}\index{reduce\_set\_to\_equal\_distribution\_of\_classes() (in module ucf)@\spxentry{reduce\_set\_to\_equal\_distribution\_of\_classes()}\spxextra{in module ucf}}

\begin{fulllineitems}
\phantomsection\label{\detokenize{api/ucf.reduce_set_to_equal_distribution_of_classes:ucf.reduce_set_to_equal_distribution_of_classes}}\pysiglinewithargsret{\sphinxcode{\sphinxupquote{ucf.}}\sphinxbfcode{\sphinxupquote{reduce\_set\_to\_equal\_distribution\_of\_classes}}}{\emph{features\_for\_training}, \emph{targets\_for\_training}}{}
Reduces the training set to the size N = K x size of the
least frequent class

Firstly, the count of least frequent class is computed. Than a pair
with features and targets is constructed consisting of samples of
all classes. Therefore, generated pair is balanced in regards to
distribution of classes.
\begin{quote}\begin{description}
\item[{Parameters}] \leavevmode\begin{itemize}
\item {} 
\sphinxstyleliteralstrong{\sphinxupquote{features\_for\_training}} (\sphinxstyleliteralemphasis{\sphinxupquote{numpy.array}}) -- Features which will be used for generating reduced sets.

\item {} 
\sphinxstyleliteralstrong{\sphinxupquote{targets\_for\_training}} (\sphinxstyleliteralemphasis{\sphinxupquote{numpy.array}}) -- Targets which will be used for generating reduced sets.

\end{itemize}

\item[{Returns}] \leavevmode
Two separate numpy.arrays Features and targets reduced to the size of
equal to the number of samples belonging to the leas frequent class.

\item[{Return type}] \leavevmode
numpy.array

\end{description}\end{quote}

\end{fulllineitems}



\subsection{scatter\_plot\_with\_groups}
\label{\detokenize{api/ucf.scatter_plot_with_groups:scatter-plot-with-groups}}\label{\detokenize{api/ucf.scatter_plot_with_groups::doc}}\index{scatter\_plot\_with\_groups() (in module ucf)@\spxentry{scatter\_plot\_with\_groups()}\spxextra{in module ucf}}

\begin{fulllineitems}
\phantomsection\label{\detokenize{api/ucf.scatter_plot_with_groups:ucf.scatter_plot_with_groups}}\pysiglinewithargsret{\sphinxcode{\sphinxupquote{ucf.}}\sphinxbfcode{\sphinxupquote{scatter\_plot\_with\_groups}}}{\emph{coordinates}, \emph{labels}, \emph{legend\_colors}, \emph{legend\_descriptions}, \emph{save\_plot=False}, \emph{path=None}}{}
Produce scatter plot with coloration according to the labels
\begin{quote}\begin{description}
\item[{Parameters}] \leavevmode\begin{itemize}
\item {} 
\sphinxstyleliteralstrong{\sphinxupquote{coordinates}} (\sphinxstyleliteralemphasis{\sphinxupquote{numpy.array}}) -- Coordinates of points.

\item {} 
\sphinxstyleliteralstrong{\sphinxupquote{labels}} (\sphinxstyleliteralemphasis{\sphinxupquote{numpy.array}}) -- Vector indicating class membership of each point.

\item {} 
\sphinxstyleliteralstrong{\sphinxupquote{legend\_colors}} (\sphinxstyleliteralemphasis{\sphinxupquote{dict}}) -- Colors to be utilized for coloration of points.

\item {} 
\sphinxstyleliteralstrong{\sphinxupquote{legend\_descriptions}} (\sphinxstyleliteralemphasis{\sphinxupquote{dict}}) -- Labels to be utilized for description in plot legend.

\item {} 
\sphinxstyleliteralstrong{\sphinxupquote{save\_plot}} (\sphinxstyleliteralemphasis{\sphinxupquote{bool}}) -- Indication whether to save a plot. Defaults to none

\item {} 
\sphinxstyleliteralstrong{\sphinxupquote{path}} (\sphinxstyleliteralemphasis{\sphinxupquote{str}}) -- Path including the file name where to save the plot.

\end{itemize}

\item[{Returns}] \leavevmode
No explicit return.

\item[{Return type}] \leavevmode
None

\end{description}\end{quote}

\end{fulllineitems}



\section{Classes}
\label{\detokenize{index:classes}}

\begin{savenotes}\sphinxatlongtablestart\begin{longtable}[c]{\X{1}{2}\X{1}{2}}
\hline

\endfirsthead

\multicolumn{2}{c}%
{\makebox[0pt]{\sphinxtablecontinued{\tablename\ \thetable{} -- continued from previous page}}}\\
\hline

\endhead

\hline
\multicolumn{2}{r}{\makebox[0pt][r]{\sphinxtablecontinued{Continued on next page}}}\\
\endfoot

\endlastfoot

\sphinxcode{\sphinxupquote{PCA}}({[}n\_components, copy, whiten, ...{]})
&
Principal component analysis (PCA)
\\
\hline
\sphinxcode{\sphinxupquote{SVC}}({[}C, kernel, degree, gamma, coef0, ...{]})
&
C-Support Vector Classification.
\\
\hline
{\hyperref[\detokenize{api/ucf.TrainingDataSets:ucf.TrainingDataSets}]{\sphinxcrossref{\sphinxcode{\sphinxupquote{TrainingDataSets}}}}}(features\_and\_targets\_data\_set)
&
Class for generating training, validation, and testing data sets.
\\
\hline
\sphinxcode{\sphinxupquote{product}}
&
product({\color{red}\bfseries{}*}iterables, repeat=1) --\textgreater{} product object
\\
\hline
\end{longtable}\sphinxatlongtableend\end{savenotes}


\subsection{TrainingDataSets}
\label{\detokenize{api/ucf.TrainingDataSets:trainingdatasets}}\label{\detokenize{api/ucf.TrainingDataSets::doc}}\index{TrainingDataSets (class in ucf)@\spxentry{TrainingDataSets}\spxextra{class in ucf}}

\begin{fulllineitems}
\phantomsection\label{\detokenize{api/ucf.TrainingDataSets:ucf.TrainingDataSets}}\pysiglinewithargsret{\sphinxbfcode{\sphinxupquote{class }}\sphinxcode{\sphinxupquote{ucf.}}\sphinxbfcode{\sphinxupquote{TrainingDataSets}}}{\emph{features\_and\_targets\_data\_set}}{}
Bases: \sphinxcode{\sphinxupquote{object}}

Class for generating training, validation, and testing data sets.
\index{original\_data (ucf.TrainingDataSets attribute)@\spxentry{original\_data}\spxextra{ucf.TrainingDataSets attribute}}

\begin{fulllineitems}
\phantomsection\label{\detokenize{api/ucf.TrainingDataSets:ucf.TrainingDataSets.original_data}}\pysigline{\sphinxbfcode{\sphinxupquote{original\_data}}}
Original features and targets.
\begin{quote}\begin{description}
\item[{Type}] \leavevmode
pandas.DataFrame

\end{description}\end{quote}

\end{fulllineitems}

\index{indices\_of\_features (ucf.TrainingDataSets attribute)@\spxentry{indices\_of\_features}\spxextra{ucf.TrainingDataSets attribute}}

\begin{fulllineitems}
\phantomsection\label{\detokenize{api/ucf.TrainingDataSets:ucf.TrainingDataSets.indices_of_features}}\pysigline{\sphinxbfcode{\sphinxupquote{indices\_of\_features}}}
Numeric indication of position of features in \sphinxtitleref{original\_data}
\begin{quote}\begin{description}
\item[{Type}] \leavevmode
list

\end{description}\end{quote}

\end{fulllineitems}

\index{indices\_of\_targets (ucf.TrainingDataSets attribute)@\spxentry{indices\_of\_targets}\spxextra{ucf.TrainingDataSets attribute}}

\begin{fulllineitems}
\phantomsection\label{\detokenize{api/ucf.TrainingDataSets:ucf.TrainingDataSets.indices_of_targets}}\pysigline{\sphinxbfcode{\sphinxupquote{indices\_of\_targets}}}
Numeric indication of position of targets in \sphinxtitleref{original\_data}
\begin{quote}\begin{description}
\item[{Type}] \leavevmode
list

\end{description}\end{quote}

\end{fulllineitems}

\index{train\_features (ucf.TrainingDataSets attribute)@\spxentry{train\_features}\spxextra{ucf.TrainingDataSets attribute}}

\begin{fulllineitems}
\phantomsection\label{\detokenize{api/ucf.TrainingDataSets:ucf.TrainingDataSets.train_features}}\pysigline{\sphinxbfcode{\sphinxupquote{train\_features}}}
Unscaled training features.
\begin{quote}\begin{description}
\item[{Type}] \leavevmode
numpy.array

\end{description}\end{quote}

\end{fulllineitems}

\index{train\_targets (ucf.TrainingDataSets attribute)@\spxentry{train\_targets}\spxextra{ucf.TrainingDataSets attribute}}

\begin{fulllineitems}
\phantomsection\label{\detokenize{api/ucf.TrainingDataSets:ucf.TrainingDataSets.train_targets}}\pysigline{\sphinxbfcode{\sphinxupquote{train\_targets}}}
Training targets. Shuffled if desired.
\begin{quote}\begin{description}
\item[{Type}] \leavevmode
numpy.array

\end{description}\end{quote}

\end{fulllineitems}

\index{validation\_features (ucf.TrainingDataSets attribute)@\spxentry{validation\_features}\spxextra{ucf.TrainingDataSets attribute}}

\begin{fulllineitems}
\phantomsection\label{\detokenize{api/ucf.TrainingDataSets:ucf.TrainingDataSets.validation_features}}\pysigline{\sphinxbfcode{\sphinxupquote{validation\_features}}}
Validation features
\begin{quote}\begin{description}
\item[{Type}] \leavevmode
numpy.array

\end{description}\end{quote}

\end{fulllineitems}

\index{validation\_targets (ucf.TrainingDataSets attribute)@\spxentry{validation\_targets}\spxextra{ucf.TrainingDataSets attribute}}

\begin{fulllineitems}
\phantomsection\label{\detokenize{api/ucf.TrainingDataSets:ucf.TrainingDataSets.validation_targets}}\pysigline{\sphinxbfcode{\sphinxupquote{validation\_targets}}}
Validation Targets
\begin{quote}\begin{description}
\item[{Type}] \leavevmode
numpy.array

\end{description}\end{quote}

\end{fulllineitems}

\index{test\_features (ucf.TrainingDataSets attribute)@\spxentry{test\_features}\spxextra{ucf.TrainingDataSets attribute}}

\begin{fulllineitems}
\phantomsection\label{\detokenize{api/ucf.TrainingDataSets:ucf.TrainingDataSets.test_features}}\pysigline{\sphinxbfcode{\sphinxupquote{test\_features}}}
Testing features
\begin{quote}\begin{description}
\item[{Type}] \leavevmode
numpy.array

\end{description}\end{quote}

\end{fulllineitems}

\index{test\_targets (ucf.TrainingDataSets attribute)@\spxentry{test\_targets}\spxextra{ucf.TrainingDataSets attribute}}

\begin{fulllineitems}
\phantomsection\label{\detokenize{api/ucf.TrainingDataSets:ucf.TrainingDataSets.test_targets}}\pysigline{\sphinxbfcode{\sphinxupquote{test\_targets}}}
Testing targets
\begin{quote}\begin{description}
\item[{Type}] \leavevmode
numpy.array

\end{description}\end{quote}

\end{fulllineitems}

\index{scaled\_train\_features (ucf.TrainingDataSets attribute)@\spxentry{scaled\_train\_features}\spxextra{ucf.TrainingDataSets attribute}}

\begin{fulllineitems}
\phantomsection\label{\detokenize{api/ucf.TrainingDataSets:ucf.TrainingDataSets.scaled_train_features}}\pysigline{\sphinxbfcode{\sphinxupquote{scaled\_train\_features}}}
Train features scaled to mean zero and unit variance. Shuffled
if desired.
\begin{quote}\begin{description}
\item[{Type}] \leavevmode
numpy.array

\end{description}\end{quote}

\end{fulllineitems}

\index{scaled\_validation\_features (ucf.TrainingDataSets attribute)@\spxentry{scaled\_validation\_features}\spxextra{ucf.TrainingDataSets attribute}}

\begin{fulllineitems}
\phantomsection\label{\detokenize{api/ucf.TrainingDataSets:ucf.TrainingDataSets.scaled_validation_features}}\pysigline{\sphinxbfcode{\sphinxupquote{scaled\_validation\_features}}}
Validation features scaled to mean zero and unit variance.
\begin{quote}\begin{description}
\item[{Type}] \leavevmode
Numpy Array

\end{description}\end{quote}

\end{fulllineitems}

\index{scaled\_test\_features (ucf.TrainingDataSets attribute)@\spxentry{scaled\_test\_features}\spxextra{ucf.TrainingDataSets attribute}}

\begin{fulllineitems}
\phantomsection\label{\detokenize{api/ucf.TrainingDataSets:ucf.TrainingDataSets.scaled_test_features}}\pysigline{\sphinxbfcode{\sphinxupquote{scaled\_test\_features}}}
Test features scaled to mean zero and unit variance.
\begin{quote}\begin{description}
\item[{Type}] \leavevmode
Numpy Array

\end{description}\end{quote}

\end{fulllineitems}

\subsubsection*{Methods Summary}


\begin{savenotes}\sphinxatlongtablestart\begin{longtable}[c]{\X{1}{2}\X{1}{2}}
\hline

\endfirsthead

\multicolumn{2}{c}%
{\makebox[0pt]{\sphinxtablecontinued{\tablename\ \thetable{} -- continued from previous page}}}\\
\hline

\endhead

\hline
\multicolumn{2}{r}{\makebox[0pt][r]{\sphinxtablecontinued{Continued on next page}}}\\
\endfoot

\endlastfoot

{\hyperref[\detokenize{api/ucf.TrainingDataSets:ucf.TrainingDataSets.compute_mean_and_standard_deviation}]{\sphinxcrossref{\sphinxcode{\sphinxupquote{compute\_mean\_and\_standard\_deviation}}}}}()
&
Computation of mean and standard deviation of features in the training data set.
\\
\hline
{\hyperref[\detokenize{api/ucf.TrainingDataSets:ucf.TrainingDataSets.get_scaled_features}]{\sphinxcrossref{\sphinxcode{\sphinxupquote{get\_scaled\_features}}}}}()
&
Convenience method to return scaled features.
\\
\hline
{\hyperref[\detokenize{api/ucf.TrainingDataSets:ucf.TrainingDataSets.get_targets}]{\sphinxcrossref{\sphinxcode{\sphinxupquote{get\_targets}}}}}()
&
Convenience method to return targets and features
\\
\hline
{\hyperref[\detokenize{api/ucf.TrainingDataSets:ucf.TrainingDataSets.make_training_data}]{\sphinxcrossref{\sphinxcode{\sphinxupquote{make\_training\_data}}}}}(train\_size, validation\_size)
&
Make features and targets
\\
\hline
{\hyperref[\detokenize{api/ucf.TrainingDataSets:ucf.TrainingDataSets.scale_features}]{\sphinxcrossref{\sphinxcode{\sphinxupquote{scale\_features}}}}}()
&
Standardize features in such manner that their mean is centered to zero, and unit of measurement is set to variance.
\\
\hline
{\hyperref[\detokenize{api/ucf.TrainingDataSets:ucf.TrainingDataSets.shuffle}]{\sphinxcrossref{\sphinxcode{\sphinxupquote{shuffle}}}}}()
&
Shuffle scaled features and unscaled targets for training
\\
\hline
\end{longtable}\sphinxatlongtableend\end{savenotes}
\subsubsection*{Methods Documentation}
\index{compute\_mean\_and\_standard\_deviation() (ucf.TrainingDataSets method)@\spxentry{compute\_mean\_and\_standard\_deviation()}\spxextra{ucf.TrainingDataSets method}}

\begin{fulllineitems}
\phantomsection\label{\detokenize{api/ucf.TrainingDataSets:ucf.TrainingDataSets.compute_mean_and_standard_deviation}}\pysiglinewithargsret{\sphinxbfcode{\sphinxupquote{compute\_mean\_and\_standard\_deviation}}}{}{}
Computation of mean and standard deviation of features
in the training data set.
\begin{quote}\begin{description}
\item[{Returns}] \leavevmode
No explicit return.

\item[{Return type}] \leavevmode
None

\end{description}\end{quote}
\subsubsection*{Notes}

Values are stored in the 'features\_mean' and
'features\_standard\_deviation' attribute of the class.

\end{fulllineitems}

\index{get\_scaled\_features() (ucf.TrainingDataSets method)@\spxentry{get\_scaled\_features()}\spxextra{ucf.TrainingDataSets method}}

\begin{fulllineitems}
\phantomsection\label{\detokenize{api/ucf.TrainingDataSets:ucf.TrainingDataSets.get_scaled_features}}\pysiglinewithargsret{\sphinxbfcode{\sphinxupquote{get\_scaled\_features}}}{}{}
Convenience method to return scaled features.
\begin{quote}\begin{description}
\item[{Returns}] \leavevmode
\begin{itemize}
\item {} 
\sphinxstylestrong{scaled\_train\_features} (\sphinxstyleemphasis{numpy.array}) -- Scaled features for training.

\item {} 
\sphinxstylestrong{scaled\_validation\_features} (\sphinxstyleemphasis{numpy.array}) -- Scaled features for validation.

\item {} 
\sphinxstylestrong{scaled\_test\_features} (\sphinxstyleemphasis{numpy.array}) -- Scaled features for testing.

\end{itemize}


\end{description}\end{quote}

\end{fulllineitems}

\index{get\_targets() (ucf.TrainingDataSets method)@\spxentry{get\_targets()}\spxextra{ucf.TrainingDataSets method}}

\begin{fulllineitems}
\phantomsection\label{\detokenize{api/ucf.TrainingDataSets:ucf.TrainingDataSets.get_targets}}\pysiglinewithargsret{\sphinxbfcode{\sphinxupquote{get\_targets}}}{}{}
Convenience method to return targets and features
\begin{quote}\begin{description}
\item[{Returns}] \leavevmode
\begin{itemize}
\item {} 
\sphinxstylestrong{train\_targets} (\sphinxstyleemphasis{numpy.array}) -- Targets for training.

\item {} 
\sphinxstylestrong{validation\_targets} (\sphinxstyleemphasis{numpy.array}) -- Targets for validation.

\item {} 
\sphinxstylestrong{test\_targets} (\sphinxstyleemphasis{numpy.array}) -- Targets for testing.

\end{itemize}


\end{description}\end{quote}

\end{fulllineitems}

\index{make\_training\_data() (ucf.TrainingDataSets method)@\spxentry{make\_training\_data()}\spxextra{ucf.TrainingDataSets method}}

\begin{fulllineitems}
\phantomsection\label{\detokenize{api/ucf.TrainingDataSets:ucf.TrainingDataSets.make_training_data}}\pysiglinewithargsret{\sphinxbfcode{\sphinxupquote{make\_training\_data}}}{\emph{train\_size}, \emph{validation\_size}}{}
Make features and targets
\begin{quote}\begin{description}
\item[{Parameters}] \leavevmode\begin{itemize}
\item {} 
\sphinxstyleliteralstrong{\sphinxupquote{train\_size}} (\sphinxstyleliteralemphasis{\sphinxupquote{int}}) -- Proportion of the training set.

\item {} 
\sphinxstyleliteralstrong{\sphinxupquote{validation\_size}} (\sphinxstyleliteralemphasis{\sphinxupquote{int}}) -- Proportion of the validation set.

\end{itemize}

\end{description}\end{quote}
\subsubsection*{Notes}

Size of the testing set is determined implicitly.

\end{fulllineitems}

\index{scale\_features() (ucf.TrainingDataSets method)@\spxentry{scale\_features()}\spxextra{ucf.TrainingDataSets method}}

\begin{fulllineitems}
\phantomsection\label{\detokenize{api/ucf.TrainingDataSets:ucf.TrainingDataSets.scale_features}}\pysiglinewithargsret{\sphinxbfcode{\sphinxupquote{scale\_features}}}{}{}
Standardize features in such manner that their mean is centered
to zero, and unit of measurement is set to variance.
\begin{quote}\begin{description}
\item[{Returns}] \leavevmode
Standardized features are placed inside appropriate
attributes of the class.

\item[{Return type}] \leavevmode
numpy.array

\end{description}\end{quote}

\end{fulllineitems}

\index{shuffle() (ucf.TrainingDataSets method)@\spxentry{shuffle()}\spxextra{ucf.TrainingDataSets method}}

\begin{fulllineitems}
\phantomsection\label{\detokenize{api/ucf.TrainingDataSets:ucf.TrainingDataSets.shuffle}}\pysiglinewithargsret{\sphinxbfcode{\sphinxupquote{shuffle}}}{}{}
Shuffle scaled features and unscaled targets for training
\subsubsection*{Notes}

Only scaled training features and targets are shuffled. Validation,
and test data sets are not shuffled.

\end{fulllineitems}


\end{fulllineitems}



\section{Class Inheritance Diagram}
\label{\detokenize{index:class-inheritance-diagram}}
\sphinxincludegraphics[]{inheritance-26805a6a50aa09a94e3e8ec84ce8efe11d4fbb3c.pdf}


\chapter{Indices and tables}
\label{\detokenize{index:indices-and-tables}}\begin{itemize}
\item {} 
\DUrole{xref,std,std-ref}{genindex}

\item {} 
\DUrole{xref,std,std-ref}{modindex}

\item {} 
\DUrole{xref,std,std-ref}{search}

\end{itemize}


\renewcommand{\indexname}{Python Module Index}
\begin{sphinxtheindex}
\let\bigletter\sphinxstyleindexlettergroup
\bigletter{u}
\item\relax\sphinxstyleindexentry{ucf}\sphinxstyleindexpageref{index:\detokenize{module-ucf}}
\end{sphinxtheindex}

\renewcommand{\indexname}{Index}
\printindex
\end{document}